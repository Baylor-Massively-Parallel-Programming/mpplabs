\documentclass{article}

\title{Lab 1}
\author{Keith Evan Schubert}

\begin{document}
\maketitle

\section{Objective}
The purpose of this lab is to get you familiar with using the CUDA API by implementing a simple vector addition kernel and its associated setup code.  Note: This is a simple but essential exercise. Please write out the code and do not copy it from other examples or lecture slides. That process is most important.

\section{Activity}

\begin{enumerate}
\item Login to kodiak.  cd to your mpplabs directory and type \verb1git pull1.
\item Edit the file \verb1<lab-directory>/main.cu1 to implement the following where indicated:
	\begin{enumerate}
	\item Allocate device memory
	\item Copy host memory to device
	\item Initialize thread block and kernel grid dimensions and invoke CUDA kernel
	\item Copy results from device to host
	\item Free device memory
	\end{enumerate}	
\item Edit the file <lab-directory>/kernel.cu to implement the vector addition kernel code.
\item Compile and test your code.  Try it for several sizes, say 1k, 10k, 100k, 1M.  How does the time change?  Does each part change the same?
\begin{verbatim}
	    cd <lab-directory>
	    make
	    nano vecadd.sh                 #add the following
	   		 ~/,lab-directory>/vecadd      # Uses the default vector size
	   		 ~/,lab-directory>/vecadd <m>  # Uses vectors of size m
	    mpprun vecadd.sh
\end{verbatim}
\end{enumerate}

\section{Turn in}
Upload a report that includes the output with analysis of the time complexity scaling, main.cu, and kernel.cu  to the course Canvas site.


\section{Going Further}
Try this same thing for changing an image to grayscale.

\end{document} 